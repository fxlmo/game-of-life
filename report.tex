%%=====================================================================================
%%
%%       Filename:  report.tex
%%
%%    Description:  Report for our project
%%
%%        Version:  1.0
%%        Created:  02/12/19
%%       Revision:  none
%%
%%         Author:  YOUR NAME (), 
%%   Organization:  
%%      Copyright:  Copyright (c) 2019, YOUR NAME
%%
%%          Notes:  
%%
%%=====================================================================================
% Preamble {{{
\documentclass[10pt,a4paper,dvipsnames,cmyk]{scrartcl}
\usepackage[english]{babel}
\typearea{12}
%}}}

% Set indentation and line skip for paragraph {{{
\setlength{\parskip}{1em}
\usepackage[margin=2cm]{geometry}
\addtolength{\textheight}{-1in}
\setlength{\headsep}{.5in}
% }}}

\usepackage{hhline} 
\usepackage{mathtools} 
\usepackage[T1]{fontenc}
\usepackage[utf8]{inputenc}

% Headers setup {{{
\usepackage{fancyhdr}
\pagestyle{fancy}
\lhead{Game of Life}
\rhead{Josh Felmeden, Antoine Ritz}
\usepackage{hyperref} 
% }}}

% Listings {{{
\usepackage[]{listings,xcolor} 
\lstset
{
    breaklines=true,
    tabsize=3,
    showstringspaces=false
}

\definecolor{lstgrey}{rgb}{0.05,0.05,0.05}
\usepackage{listings}
\makeatletter
\lstset{language=[Visual]Basic,
    backgroundcolor=\color{lstgrey},
    frame=single,
    xleftmargin=0.7cm,
    frame=tlbr, framesep=0.2cm, framerule=0pt,
    basicstyle=\lst@ifdisplaystyle\color{white}\footnotesize\ttfamily\else\color{black}\footnotesize\ttfamily\fi,
    captionpos=b,
    tabsize=2,
    keywordstyle=\color{Magenta}\bfseries,
    identifierstyle=\color{Cyan},
    stringstyle=\color{Yellow},
    commentstyle=\color{Gray}\itshape
}
\makeatother
\renewcommand{\familydefault}{\sfdefault}
% }}}


% Other packages {{{
\usepackage{pgfplots}
\pgfplotsset{compat=1.16}
\usepackage{tikz}
\usepackage{hhline}
\usepackage{multicol}
\usepackage{needspace}
\usepackage{tcolorbox}
\usepackage{soul}
\usepackage{CormorantGaramond} 
%% The font package uses mweights.sty which has som issues with the
%% \normalfont command. The following two lines fixes this issue.
\let\oldnormalfont\normalfont
\def\normalfont{\oldnormalfont\mdseries}
\usepackage{textcomp}
\usepackage{amsmath} 
\usepackage{booktabs} 
\usepackage{tcolorbox} 
\usepackage[symbol]{footmisc} 
\renewcommand{\thefootnote}{\fnsymbol{footnote}}
\renewcommand{\familydefault}{\sfdefault}
% }}}
\usepackage{caption,subcaption}

% Title {{{
\title{The Game of Life}
\subtitle{Concurrent Computing Coursework}
\author{Josh Felmeden, NK18044 \\ Antoine Ritz, EV18263}
% }}}

\usepackage{filecontents}
\begin{filecontents*}{16x16.csv}
threads,single,divide-conquer,division,coop
16x16x2-12,0.033483939,0.048826398,0.059129326,0.028504678
16x16x4-12,0.032854597,0.046599358,0.057434406,0.025045960
16x16x8-12,0.040454184,0.061394925,0.065987021,0.016747854
\end{filecontents*}
\begin{filecontents*}{128x128.csv}
threads,single,divide-conquer,division,coop
128x128x2-12,2.035618015,1.733091481,1.688893352,1.066302327
128x128x4-12,2.032771685,1.293476092,1.253884265,0.560477459
128x128x8-12,2.035792169,1.011711902,1.002440810,0.335816265
\end{filecontents*}
\begin{filecontents*}{512x512.csv}
threads,single,divide-conquer,division,coop
512x512x2-12,32.203317370,26.492271674,27.127623261,10.434325929
512x512x4-12,32.129000733,18.341708142,18.868956173,6.353856677
512x512x8-12,32.140070313,14.104958044,15.309905872,4.286390864
\end{filecontents*}
\begin{filecontents*}{average.csv}
solution,average
Single,135.048
Divide-conquer,90.460
Division,90.832
Cooperative,41.679
\end{filecontents*}
\begin{document}
\maketitle
\begin{multicols}{2}

\section*{Functionality and Design}%
\label{sec:func-and-design}
Our solution was built up by initially creating a single threaded solution
to the problem. This version iterates through the board bitwise, and for
each bit gathers all the `neighbours' for the cells (the 8 directly
adjacent cells). From this, the logic is applied and the cell is updated
if necessary. This is repeated for the desired number of turns.

From this, we created a multi-threaded solution. We split the board up
into strips and passed each strip to a worker. However, each worker would
also need information from the lines directly above and below its strip of
cells (called \textit{halo lines}). We decided to pass these halo lines
wrapped around the strips, so that the workers are able to calculate each
cell correctly. Once they have completed their strip, they return it to
the \texttt{distributor} function. The function reconstructs the world and begins
the process again for the desired number of turns.

One problem that we ran into was that we were passing the world by means
of pointer. This led to problems due to premature changes being applied to
the board. To solve this, we used two channels: \texttt{inputChans} and
\texttt{outChans} (an array of channels with one channel for each worker)
to pass the board to the workers.

The processing of the program is currently unable to be cancelled, and
therefore we added the ability to quit, pause processing, and show the
current state of the board with key presses. Alongside this, we also
implemented an output of the number of alive cells every two seconds using
a \textit{ticker}. 

Following this, we added the ability to allow the number of workers to
support all multiples of two, rather than powers of two alone. Initially,
this proved difficult, since at least one worker would receive a smaller
strip than the others, and consequently meant that some workers would
finish sooner than others. This resulted in a \textbf{deadlock}, because
the world reconstructing function expected all strips to be of equal size,
and therefore it attempts to read a strip of equal size for every worker.
However, the final worker returns a smaller strip and so the
\texttt{distributor} function is left waiting for more bits to arrive on
the channel indefinitely. To resolve this, we made each worker apart
from the final one work on the same number of lines, and the final worker
simply had the remainder. Then, to reconstruct the world, we processed the
final worker output separately from the rest.

Finally, passing the entire world between each turn is time consuming,
because in reality, the only information that needs to be passed between
the workers are the \textit{halo lines}. Additionally, there is no reason
to reconstruct the entire world each turn.

Implementing this proved to be problematic, as sometimes the workers would
get out of sync with one another. To rectify this, we implemented a
\textit{master} thread, which served as a hard limit on the speed of the
other threads to ensure that the threads would not exceed one another. We
passed in a new structure called \texttt{flags} with a variable called
\texttt{masterTurns} via pointer, which contains the current turn of the
master thread.

We were unable to implement the user interaction via key press for this
implementation, however, the solution does implement the required
computations.
\end{multicols}

\newpage
\section*{Tests, Experiments, and Critical Analysis}%
\label{sec:experiment-and-analysis}
%TODO: Text goes here

\subsection*{Stage 1a --- Single Thread}%
\label{sub:single-thread}

\begin{table}[!htb]
\caption{Benchmark comparison for Stage 1a}
\begin{center}
    \begin{tabular}{|c|c|c|c|}
        \hline
        \textbf{Benchmark} & \textbf{Baseline result} (ns/100 turns) &
        \textbf{Our result} & \textbf{\% Difference} \\ \hhline{|=|=|=|=|}
        128x128x2-12 & $73689886$ & $2021400726$ & $36\%$ \\ \hline
        128x128x4-12 & $538915394$ & $2010389496$ & $26\%$ \\ \hline
        128x128x8-12 & $261671491$ & $2009465802$ & $17\%$ \\ \hline
    \end{tabular}
\end{center}
\end{table}

\begin{table}[!htb]
\caption{CPU usage comparison for Stage 1a}
\begin{center}
    \begin{tabular}{|c|c|c|c|}
        \hline
        \textbf{Benchmark} & \textbf{Baseline CPU usage} &
        \textbf{Our CPU usage} & \textbf{\% Difference} \\ \hhline{|=|=|=|=|}
        128x128x2-12 & $185\%$ & $100\%$ & $185\%$ \\ \hline
        128x128x4-12 & $298\%$ & $100\%$ & $298\%$ \\ \hline
        128x128x8-12 & $425\%$ & $100\%$ & $425\%$ \\ \hline
    \end{tabular}
\end{center}
\end{table}


% Average bench: $135.048$s

This solution is the slowest because of the lack of multi-threading. The
average bench is $135.048$s.

\subsection*{Stage 1b --- Divide and Conquer}%
\label{sub:divide-conquer}
\begin{table}[!htb]
\caption{Benchmark comparison for Stage 1b}
\begin{center}
    \begin{tabular}{|c|c|c|c|}
        \hline
        \textbf{Benchmark} & \textbf{Baseline result} (ns/100 turns) &
        \textbf{Our result} & \textbf{\% Difference} \\ \hhline{|=|=|=|=|}
        128x128x2-12 & $736946012$ & $1704893591$ & $43\%$ \\ \hline
        128x128x4-12 & $537751669$ & $1273465510$ & $42\%$ \\ \hline
        128x128x8-12 & $362735057$ & $1004595662$ & $36\%$ \\ \hline
    \end{tabular}
\end{center}
\end{table}
\begin{table}[!htb]
\caption{CPU usage comparison for Stage 1b}
\begin{center}
    \begin{tabular}{|c|c|c|c|}
        \hline
        \textbf{Benchmark} & \textbf{Baseline CPU usage} &
        \textbf{Our CPU usage} & \textbf{\% Difference} \\ \hhline{|=|=|=|=|}
        128x128x2-12 & $185\%$ & $188\%$ & $98\%$ \\ \hline
        128x128x4-12 & $298\%$ & $271\%$ & $109\%$ \\ \hline
        128x128x8-12 & $425\%$ & $348\%$ & $122\%$ \\ \hline
    \end{tabular}
\end{center}
\end{table}

% Average bench: $90.460s$

By dividing the work up between the workers, this solution improves
greatly on the single threaded solution, with an average benchmark of
$90.460$s.

\newpage
\subsection*{Stage 2a --- User Interaction}%
\label{sub:user-interaction}

% \begin{center}
%     \begin{tabular}{|c|c|c|c|}
%         \hline
%         \textbf{Benchmark} & \textbf{Baseline result} (ns/100 turns) &
%         \textbf{Our result} & \textbf{\% Difference} \\ \hhline{|=|=|=|=|}
%         128x128x2-12 & $736121962$ & $1699574829$ & $43\%$ \\ \hline
%         128x128x4-12 & $537944647$ & $1256970618$ & $42\%$ \\ \hline
%         128x128x8-12 & $361777916$ & $994380759$ & $36\%$ \\ \hline
%     \end{tabular}
% \end{center}
% \begin{center}
%     \begin{tabular}{|c|c|c|c|}
%         \hline
%         \textbf{Benchmark} & \textbf{Baseline CPU usage} &
%         \textbf{Our CPU usage} & \textbf{\% Difference} \\ \hhline{|=|=|=|=|}
%         128x128x2-12 & $185\%$ & $188\%$ & $98\%$ \\ \hline
%         128x128x4-12 & $296\%$ & $273\%$ & $108\%$ \\ \hline
%         128x128x8-12 & $426\%$ & $343\%$ & $124\%$ \\ \hline
%     \end{tabular}
% \end{center}
% Average bench: $88.539$s

The benchmark results of stages 2a, 2b, and 3 are almost identical to that
of stage 1b due to the lack of changes to the processing of the actual
Game of Life logic. There are minor time increases for each stage due to
additional constant processing.  In this solution, there is a go routine that checks for a key input
alongside the computation of the Game of Life logic which impacts the
benchmark slightly. The average benchmark for this solution is $88.539$s.

\subsection*{Stage 2b --- Periodic Events}%
\label{sub:periodic-events}
% \begin{center}
%     \begin{tabular}{|c|c|c|c|}
%         \hline
%         \textbf{Benchmark} & \textbf{Baseline result} (ns/100 turns) &
%         \textbf{Our result} & \textbf{\% Difference} \\ \hhline{|=|=|=|=|}
%         128x128x2-12 & $735806211$ & $1718965842$ & $42\%$ \\ \hline
%         128x128x4-12 & $532913881$ & $1277891763$ & $41\%$ \\ \hline
%         128x128x8-12 & $361377377$ & $998142692$ & $36\%$ \\ \hline
%     \end{tabular}
% \end{center}
% \begin{center}
%     \begin{tabular}{|c|c|c|c|}
%         \hline
%         \textbf{Benchmark} & \textbf{Baseline CPU usage} &
%         \textbf{Our CPU usage} & \textbf{\% Difference} \\ \hhline{|=|=|=|=|}
%         128x128x2-12 & $185\%$ & $190\%$ & $97\%$ \\ \hline
%         128x128x4-12 & $299\%$ & $274\%$ & $109\%$ \\ \hline
%         128x128x8-12 & $424\%$ & $347\%$ & $122\%$ \\ \hline
%     \end{tabular}
% \end{center}

% Average bench: $94.175$s

For this solution, there is a \texttt{ticker} that outputs the number of
alive cells every 2 seconds. The impact on the overall runtime of the
program is minimal, with the average benchmark being $94.175$s.

\subsection*{Stage 3 --- Division of Work}%
\label{sub:division-of-work}
% \begin{center}
%     \begin{tabular}{|c|c|c|c|}
%         \hline
%         \textbf{Benchmark} & \textbf{Baseline result} (ns/100 turns) &
%         \textbf{Our result} & \textbf{\% Difference} \\ \hhline{|=|=|=|=|}
%         128x128x2-12 & $736951655$ & $1730991762$ & $42\%$ \\ \hline
%         128x128x4-12 & $537184229$ & $1264782083$ & $42\%$ \\ \hline
%         128x128x8-12 & $363069928$ & $987918516$ & $36\%$ \\ \hline
%     \end{tabular}
% \end{center}
% \begin{center}
%     \begin{tabular}{|c|c|c|c|}
%         \hline
%         \textbf{Benchmark} & \textbf{Baseline CPU usage} &
%         \textbf{Our CPU usage} & \textbf{\% Difference} \\ \hhline{|=|=|=|=|}
%         128x128x2-12 & $184\%$ & $190\%$ & $96\%$ \\ \hline
%         128x128x4-12 & $299\%$ & $274\%$ & $109\%$ \\ \hline
%         128x128x8-12 & $425\%$ & $347\%$ & $122\%$ \\ \hline
%     \end{tabular}
% \end{center}

% Average bench: $90.832$s

This implementation is the same as stage 2b in the case of the benchmarks,
since the benchmark does not test for processes with threads that are not
powers of two. The average benchmark for this solution is $90.832$s; a
minor improvement on stage 2b that is simply a result of the normal
variation in program runtime.

\subsection*{Stage 4 --- Cooperative Problem Solving}%
\label{sub:coop-solving}
\begin{table}[!htb]
\caption{Benchmark comparison for Stage 4}
\begin{center}
    \begin{tabular}{|c|c|c|c|}
        \hline
        \textbf{Benchmark} & \textbf{Baseline result} (ns/100 turns) &
        \textbf{Our result} & \textbf{\% Difference} \\ \hhline{|=|=|=|=|}
        128x128x2-12 & $735620822$ & $716510865$ & $102\%$ \\ \hline
        128x128x4-12 & $530949831$ & $483555762$ & $109\%$ \\ \hline
        128x128x8-12 & $361530673$ & $333119799$ & $108\%$ \\ \hline
    \end{tabular}
\end{center}
\end{table}
\begin{table}[!htb]
\caption{CPU usage for Stage 4}
\begin{center}
    \begin{tabular}{|c|c|c|c|}
        \hline
        \textbf{Benchmark} & \textbf{Baseline CPU usage} &
        \textbf{Our CPU usage} & \textbf{\% Difference} \\ \hhline{|=|=|=|=|}
        128x128x2-12 & $185\%$ & $289\%$ & $64\%$ \\ \hline
        128x128x4-12 & $299\%$ & $425\%$ & $70\%$ \\ \hline
        128x128x8-12 & $425\%$ & $667\%$ & $63\%$ \\ \hline
    \end{tabular}
\end{center}
\end{table}



% Average bench: $41.679$s

The final solution is significantly faster than all previous
implementations by drastically reducing the necessary computations. The
average benchmark is reduced by over $50\%$ to only $41.679$s.

%TODO INSERT GRAPHS


\begin{samepage}
\subsection*{Conclusions}%
\label{sub:Conclusions}
For smaller image sizes (Figure 1a), the single thread solution
outperforms the initial divide and conquer method. This is because of the
large overhead cost of splitting the image up and reconstructing the world
after each turn.  However, for larger images (Figures 1b and 1c), the
divide and conquer algorithm does improve on the times set by the single
thread, because the cost of reconstructing the world and splitting the
threads is constant and does not rely on the size of the image.

In general, the performance of the divide and conquer algorithm and the
division of work algorithm is largely the same, especially for larger
images. This is due to the fact that the only major difference between the
two algorithms is that division of work algorithm allows the splitting of
threads by multiples of 2, unlike the divide and conquer algorithm which
splits the threads by powers of 2. This disparity between the two algorithms
does not affect the performance of the algorithm and explains the minimal
difference in the performance of the two solutions.

The cooperative solving solution is consistently faster than any other
algorithm. The cooperative solving algorithm is always faster than the
stage 3 solution because the cooperative solving algorithm does not
reconstruct the entire world after each turn, and instead passes the halo
lines to the worker above or below it. Another point where the cooperative
solution saves time against the stage 3 implementation is the fact that
there is no need to calculate the splits on every turn, since each worker
retains the information of its split after every turn. Additionally, the
workload of each worker remains almost exactly the same as the workers in
the stage 3 implementation.

The cooperative solution vastly outperforms the single thread solution due
to the ability to divide the work among the separate threads. The single
thread must procedurally compute each bit of the world, while the
cooperative solution is able to have each thread computing a bit at the
same time.
\end{samepage}

\pgfplotstableread[col sep=comma]{16x16.csv}\datatable
\begin{figure}[!htb]
    \begin{subfigure}{0.58\textwidth}
\begin{minipage}[b]{\textwidth}
\resizebox{\textwidth}{!}{%
\begin{tikzpicture}
\begin{axis}[
    legend style={
        at={(1,1)},
    anchor=north west,at={(axis description cs:1.1,0.7)}},
    xtick=data,
    title style={yshift=1.1ex,},
    title={\textbf{Comparison for 16x16 image}},
    xticklabels from table={\datatable}{threads},
    grid=both,
    xlabel={Picture size, threads and turns},
    ylabel={Seconds per operation},
]
\addplot table [col sep=comma, x expr=\coordindex, y=single, col sep=comma] {16x16.csv};
\addplot table [col sep=comma, x expr=\coordindex, y=divide-conquer, col sep=comma] {16x16.csv};
\addplot+[green] table [col sep=comma, x expr=\coordindex, y=coop, col sep=comma] {16x16.csv};
\addplot+[black] table [col sep=comma, x expr=\coordindex, y=division, col sep=comma] {16x16.csv};
    
\legend{Single, Divide and Conquer, Cooperative solving, Division of work}
\end{axis}
\end{tikzpicture}
}
\end{minipage}
\caption{16x16 seconds per operation comparison}
\end{subfigure}
\hfill
\begin{subfigure}{0.41\textwidth}
\begin{minipage}[b]{\textwidth}
\resizebox{\textwidth}{!}{%
\pgfplotstableread[col sep=comma]{128x128.csv}\datatable
\begin{tikzpicture}
\begin{axis}[
    xtick=data,
    title style={yshift=1.1ex,},
    title={\textbf{Comparison for 128x128 image}},
    xticklabels from table={\datatable}{threads},
    grid=both,
    xlabel={Picture size, threads and turns},
    ylabel={Seconds per operation},
]
\addplot table [col sep=comma, x expr=\coordindex, y=single, col
    sep=comma] {128x128.csv};
\addplot table [col sep=comma, x expr=\coordindex, y=divide-conquer, col
    sep=comma] {128x128.csv};
\addplot+[green] table [col sep=comma, x expr=\coordindex, y=coop, col
    sep=comma] {128x128.csv};
\addplot+[black] table [col sep=comma, x expr=\coordindex, y=division, col
    sep=comma] {128x128.csv};
\end{axis}
\end{tikzpicture}
}
\end{minipage}
\caption{128x128 seconds per operation comparison}
\end{subfigure}

\pgfplotstableread[col sep=comma]{512x512.csv}\datatable
\begin{subfigure}{0.58\textwidth}
\begin{minipage}[b]{0.49\textwidth}
\begin{tikzpicture}
\begin{axis}[
    xtick=data,
    title style={yshift=1.1ex,},
    title={\textbf{Comparison for 512x512 image}},
    xticklabels from table={\datatable}{threads},
    grid=both,
    xlabel={Picture size, threads and turns},
    ylabel={Seconds per operation},
]
\addplot table [col sep=comma, x expr=\coordindex, y=single, col
    sep=comma] {512x512.csv};
\addplot table [col sep=comma, x expr=\coordindex, y=divide-conquer, col
    sep=comma] {512x512.csv};
\addplot+[green] table [col sep=comma, x expr=\coordindex, y=coop, col
    sep=comma] {512x512.csv};
\addplot+[black] table [col sep=comma, x expr=\coordindex, y=division, col
    sep=comma] {512x512.csv};
\end{axis}
\end{tikzpicture}
\end{minipage}
\caption{512x512 seconds per operation comparison}
\end{subfigure}
\hfill
\pgfplotstableread[col sep=comma]{average.csv}\datatable
\begin{subfigure}{0.58\textwidth}
\begin{minipage}[b]{0.49\textwidth}
\begin{tikzpicture}
\begin{axis}[
    xtick=data,
    title style={yshift=1.1ex,},
    title={\textbf{Comparison for average benchmarks}},
    xticklabels from table={\datatable}{solution},
    grid=both,
    xlabel={Implementation},
    ylabel={Average benchmark (s)},
]
\addplot+[purple] table [col sep=comma, x expr=\coordindex, y=average, col
    sep=comma] {average.csv};
\end{axis}
\end{tikzpicture}
\end{minipage}
\caption{Average benchmarks for each implementation}
\end{subfigure}
\caption{Comparison graphs}
\end{figure}

\end{document}
