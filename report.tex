%%=====================================================================================
%%
%%       Filename:  report.tex
%%
%%    Description:  Report for our project
%%
%%        Version:  1.0
%%        Created:  02/12/19
%%       Revision:  none
%%
%%         Author:  YOUR NAME (), 
%%   Organization:  
%%      Copyright:  Copyright (c) 2019, YOUR NAME
%%
%%          Notes:  
%%
%%=====================================================================================
% Preamble {{{
\documentclass[11pt,a4paper,twocolumn,dvipsnames,cmyk]{scrartcl}
\usepackage[english]{babel}
\typearea{12}
% }}}

% Set indentation and line skip for paragraph {{{
\setlength{\parskip}{1em}
\usepackage[margin=2cm]{geometry}
\addtolength{\textheight}{-1in}
\setlength{\headsep}{.5in}
% }}}

\usepackage{hhline} 
\usepackage{mathtools} 
\usepackage[T1]{fontenc}

% Headers setup {{{
\usepackage{fancyhdr}
\pagestyle{fancy}
\lhead{Concurrent Computing Coursework}
\rhead{Josh Felmeden, Antoine Ritz}
\usepackage{hyperref} 
% }}}

% Listings {{{
\usepackage[]{listings,xcolor} 
\lstset
{
    breaklines=true,
    tabsize=3,
    showstringspaces=false
}

\definecolor{lstgrey}{rgb}{0.05,0.05,0.05}
\usepackage{listings}
\makeatletter
\lstset{language=[Visual]Basic,
    backgroundcolor=\color{lstgrey},
    frame=single,
    xleftmargin=0.7cm,
    frame=tlbr, framesep=0.2cm, framerule=0pt,
    basicstyle=\lst@ifdisplaystyle\color{white}\footnotesize\ttfamily\else\color{black}\footnotesize\ttfamily\fi,
    captionpos=b,
    tabsize=2,
    keywordstyle=\color{Magenta}\bfseries,
    identifierstyle=\color{Cyan},
    stringstyle=\color{Yellow},
    commentstyle=\color{Gray}\itshape
}
\makeatother
\renewcommand{\familydefault}{\sfdefault}
% }}}


% Other packages {{{
\usepackage{needspace}
\usepackage{tcolorbox}
\usepackage{soul}
\usepackage{babel,dejavu,helvet} 
\usepackage{amsmath} 
\usepackage{booktabs} 
\usepackage{tcolorbox} 
\usepackage[symbol]{footmisc} 
\renewcommand{\thefootnote}{\fnsymbol{footnote}}
\renewcommand{\familydefault}{\sfdefault}
% }}}

% Title {{{
\title{Concurrent Computing Coursework}
\author{Josh Felmeden}
% }}}

\begin{document}
\section*{Introduction}%
\label{sec:Introduction}
We have created a program that is able to simulate the cellular automaton
developed by John Horton Conway named `The Game of Life'. The game evolution is determined by its initial state and requires no further input. Every cell interacts with its eight neighbour pixels: cells that are horizontally, vertically, or diagonally adjacent. At each matrix update in time the following transitions may occur to create the next evolution of the domain:

\begin{itemize}
    \item Any live cell with fewer than two live neighbours dies
    \item Any live cell with two or three live neighbours is unaffected
    \item Any live cell with more than three live neighbours dies
    \item Any dead cell with exactly three live neighbours becomes alive
\end{itemize}

Our program successfully computes the automaton for a given number of
moves concurrently, using multiple threads.
%TODO: Text goes here

\section*{Functionality and Design}%
\label{sec:func-and-design}
Our solution was built up by initially creating a single threaded solution
to the problem. This version iterates through the board bitwise, and for
each bit gathers all of the `neighbours' for the cells (the 8 directly
adjacent cells). From this, the logic is applied and the cell is updated
if necessary. This is repeated for the desired number of turns.

From this, we created a multi-threaded solution. We split the board up
into strips and passed each strip to a worker. However, each worker would
also need information from the lines directly above and below its strip of
cells (called \textit{halo lines}). We decided to pass these halo lines
wrapped around the strips, so that the workers are able to calculate each
cell correctly. Once they have completed their strip, they return it to
the distributor function. The function reconstructs the world and begins
the process again for the desired number of turns.
%TODO: Text goes here

\section*{Experiments and Critical Analysis}%
\label{sec:Experiments and Critical Analysis}
%TODO: Text goes here




\end{document}
