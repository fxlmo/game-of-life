%%=====================================================================================
%%
%%       Filename:  report.tex
%%
%%    Description:  Report for our project
%%
%%        Version:  1.0
%%        Created:  02/12/19
%%       Revision:  none
%%
%%         Author:  YOUR NAME (), 
%%   Organization:  
%%      Copyright:  Copyright (c) 2019, YOUR NAME
%%
%%          Notes:  
%%
%%=====================================================================================
% Preamble {{{
\documentclass[11pt,a4paper,dvipsnames,cmyk]{scrartcl}
\usepackage[english]{babel}
\typearea{12}
%}}}

% Set indentation and line skip for paragraph {{{
\setlength{\parskip}{1em}
\usepackage[margin=2cm]{geometry}
\addtolength{\textheight}{-1in}
\setlength{\headsep}{.5in}
% }}}

\usepackage{hhline} 
\usepackage{mathtools} 
\usepackage[T1]{fontenc}
\usepackage[utf8]{inputenc}

% Headers setup {{{
\usepackage{fancyhdr}
\pagestyle{fancy}
\lhead{Game of Life}
\rhead{Josh Felmeden, Antoine Ritz}
\usepackage{hyperref} 
% }}}

% Listings {{{
\usepackage[]{listings,xcolor} 
\lstset
{
    breaklines=true,
    tabsize=3,
    showstringspaces=false
}

\definecolor{lstgrey}{rgb}{0.05,0.05,0.05}
\usepackage{listings}
\makeatletter
\lstset{language=[Visual]Basic,
    backgroundcolor=\color{lstgrey},
    frame=single,
    xleftmargin=0.7cm,
    frame=tlbr, framesep=0.2cm, framerule=0pt,
    basicstyle=\lst@ifdisplaystyle\color{white}\footnotesize\ttfamily\else\color{black}\footnotesize\ttfamily\fi,
    captionpos=b,
    tabsize=2,
    keywordstyle=\color{Magenta}\bfseries,
    identifierstyle=\color{Cyan},
    stringstyle=\color{Yellow},
    commentstyle=\color{Gray}\itshape
}
\makeatother
\renewcommand{\familydefault}{\sfdefault}
% }}}


% Other packages {{{
\usepackage{pgfplots}
\pgfplotsset{compat=1.16}
\usepackage{tikz}
\usepackage{hhline}
\usepackage{multicol}
\usepackage{needspace}
\usepackage{tcolorbox}
\usepackage{soul}
\usepackage{CormorantGaramond} 
%% The font package uses mweights.sty which has som issues with the
%% \normalfont command. The following two lines fixes this issue.
\let\oldnormalfont\normalfont
\def\normalfont{\oldnormalfont\mdseries}
\usepackage{textcomp}
\usepackage{amsmath} 
\usepackage{booktabs} 
\usepackage{tcolorbox} 
\usepackage[symbol]{footmisc} 
\renewcommand{\thefootnote}{\fnsymbol{footnote}}
\renewcommand{\familydefault}{\sfdefault}
% }}}

% Title {{{
\title{The Game of Life}
\subtitle{Concurrent Computing Coursework}
\author{Josh Felmeden, NK18044 \\ Antoine Ritz, EV18263}
% }}}

\usepackage{filecontents}
\begin{filecontents*}{16x16.csv}
threads,single,divide-conquer,division,coop
16x16x2-12,33483939,48826398,59129326,28504678
16x16x4-12,32854597,46599358,57434406,25045960
16x16x8-12,40454184,61394925,65987021,16747854
\end{filecontents*}
\begin{filecontents*}{128x128.csv}
threads,single,divide-conquer,division,coop
128x128x2-12,2035618015,1733091481,1688893352,1066302327
128x128x4-12,2032771685,1293476092,1253884265,560477459
128x128x8-12,2035792169,1011711902,1002440810,335816265
\end{filecontents*}
\begin{filecontents*}{512x512.csv}
threads,single,divide-conquer,division,coop
512x512x2-12,32203317370,26492271674,27127623261,10434325929
512x512x4-12,32129000733,18341708142,18868956173,6353856677
512x512x8-12,32140070313,14104958044,15309905872,4286390864
\end{filecontents*}

\begin{document}
\maketitle
\begin{multicols}{2}

\section*{Functionality and Design}%
\label{sec:func-and-design}
Our solution was built up by initially creating a single threaded solution
to the problem. This version iterates through the board bitwise, and for
each bit gathers all the `neighbours' for the cells (the 8 directly
adjacent cells). From this, the logic is applied and the cell is updated
if necessary. This is repeated for the desired number of turns.

From this, we created a multi-threaded solution. We split the board up
into strips and passed each strip to a worker. However, each worker would
also need information from the lines directly above and below its strip of
cells (called \textit{halo lines}). We decided to pass these halo lines
wrapped around the strips, so that the workers are able to calculate each
cell correctly. Once they have completed their strip, they return it to
the \texttt{distributor} function. The function reconstructs the world and begins
the process again for the desired number of turns.

One problem that we ran into was that we were passing the world by means
of pointer. This led to problems due to premature changes being applied to
the board. To solve this, we used channels to pass the board to the
workers.

The processing of the program is currently unable to be cancelled, and
therefore we added the ability to quit, pause processing, and show the
current state of the board with key presses. Alongside this, we also
implemented an output of the number of alive cells every two seconds using
a \textit{ticker}. 

Following this, we added the ability to allow the number of workers to
support all multiples of two, rather than powers of two alone. Initially,
this proved difficult, since at least one worker would receive a smaller
strip than the others, and consequently meant that some workers would
finish sooner than others. This resulted in a \textbf{deadlock} situation,
because the world reconstructing function expected all strips to be of
equal size, and therefore it was blocked when the final strip was smaller
than expected. To work around this, we made each worker apart from the
final one work on the same number of lines, and the final worker simply
had the remainder. Then, to reconstruct the world, we processed the final
worker output separately from the rest.

Finally, passing the entire world between each turn is time consuming,
because in reality, the only information that needs to be passed between
the workers are the \textit{halo lines}. Additionally, there is no reason
to reconstruct the entire world each turn. Implementing this proved to be
problematic, as sometimes the workers would get out of sync with one
another. To rectify this, we implemented a \textit{master} thread, which
served as a hard limit on the speed of the other threads to ensure that
the threads would not exceed one another. We passed in a new structure
with \texttt{masterTurns} via pointer, which contains the current turn of
the master thread.

A further problem was to implement the key presses %I DONT KNOW HOW I'M
%GOING TO DO THIS YET


%TODO: Text goes here

\end{multicols}
\newpage

\section*{Tests, Experiments, and Critical Analysis}%
\label{sec:experiment-and-analysis}
%TODO: Text goes here

\subsection*{Stage 1a --- Single Thread}%
\label{sub:single-thread}
\begin{center}
    \begin{tabular}{|c|c|c|c|}
        \hline
        \textbf{Benchmark} & \textbf{Baseline result} (ns/100 turns) &
        \textbf{Our result} & \textbf{\% Difference} \\ \hhline{|=|=|=|=|}
        128x128x2-12 & $73689886$ & $2021400726$ & $36\%$ \\ \hline
        128x128x4-12 & $538915394$ & $2010389496$ & $26\%$ \\ \hline
        128x128x8-12 & $261671491$ & $2009465802$ & $17\%$ \\ \hline
    \end{tabular}
\end{center}

\begin{center}
    \begin{tabular}{|c|c|c|c|}
        \hline
        \textbf{Benchmark} & \textbf{Baseline CPU usage} &
        \textbf{Our CPU usage} & \textbf{\% Difference} \\ \hhline{|=|=|=|=|}
        128x128x2-12 & $185\%$ & $100\%$ & $185\%$ \\ \hline
        128x128x4-12 & $298\%$ & $100\%$ & $298\%$ \\ \hline
        128x128x8-12 & $425\%$ & $100\%$ & $425\%$ \\ \hline
    \end{tabular}
\end{center}


Average bench: $135.048$s

\subsection*{Stage 1b --- Divide and Conquer}%
\label{sub:divide-conquer}
\begin{center}
    \begin{tabular}{|c|c|c|c|}
        \hline
        \textbf{Benchmark} & \textbf{Baseline result} (ns/100 turns) &
        \textbf{Our result} & \textbf{\% Difference} \\ \hhline{|=|=|=|=|}
        128x128x2-12 & $736946012$ & $1704893591$ & $43\%$ \\ \hline
        128x128x4-12 & $537751669$ & $1273465510$ & $42\%$ \\ \hline
        128x128x8-12 & $362735057$ & $1004595662$ & $36\%$ \\ \hline
    \end{tabular}
\end{center}
\begin{center}
    \begin{tabular}{|c|c|c|c|}
        \hline
        \textbf{Benchmark} & \textbf{Baseline CPU usage} &
        \textbf{Our CPU usage} & \textbf{\% Difference} \\ \hhline{|=|=|=|=|}
        128x128x2-12 & $185\%$ & $188\%$ & $98\%$ \\ \hline
        128x128x4-12 & $298\%$ & $271\%$ & $109\%$ \\ \hline
        128x128x8-12 & $425\%$ & $348\%$ & $122\%$ \\ \hline
    \end{tabular}
\end{center}

Average bench: $90.460s$

\subsection*{Stage 2a --- User Interaction}%
\label{sub:user-interaction}
\begin{center}
    \begin{tabular}{|c|c|c|c|}
        \hline
        \textbf{Benchmark} & \textbf{Baseline result} (ns/100 turns) &
        \textbf{Our result} & \textbf{\% Difference} \\ \hhline{|=|=|=|=|}
        128x128x2-12 & $736121962$ & $1699574829$ & $43\%$ \\ \hline
        128x128x4-12 & $537944647$ & $1256970618$ & $42\%$ \\ \hline
        128x128x8-12 & $361777916$ & $994380759$ & $36\%$ \\ \hline
    \end{tabular}
\end{center}
\begin{center}
    \begin{tabular}{|c|c|c|c|}
        \hline
        \textbf{Benchmark} & \textbf{Baseline CPU usage} &
        \textbf{Our CPU usage} & \textbf{\% Difference} \\ \hhline{|=|=|=|=|}
        128x128x2-12 & $185\%$ & $188\%$ & $98\%$ \\ \hline
        128x128x4-12 & $296\%$ & $273\%$ & $108\%$ \\ \hline
        128x128x8-12 & $426\%$ & $343\%$ & $124\%$ \\ \hline
    \end{tabular}
\end{center}

Average bench: $88.539$s

\subsection*{Stage 2b --- Periodic Events}%
\label{sub:periodic-events}
\begin{center}
    \begin{tabular}{|c|c|c|c|}
        \hline
        \textbf{Benchmark} & \textbf{Baseline result} (ns/100 turns) &
        \textbf{Our result} & \textbf{\% Difference} \\ \hhline{|=|=|=|=|}
        128x128x2-12 & $735806211$ & $1718965842$ & $42\%$ \\ \hline
        128x128x4-12 & $532913881$ & $1277891763$ & $41\%$ \\ \hline
        128x128x8-12 & $361377377$ & $998142692$ & $36\%$ \\ \hline
    \end{tabular}
\end{center}
\begin{center}
    \begin{tabular}{|c|c|c|c|}
        \hline
        \textbf{Benchmark} & \textbf{Baseline CPU usage} &
        \textbf{Our CPU usage} & \textbf{\% Difference} \\ \hhline{|=|=|=|=|}
        128x128x2-12 & $185\%$ & $190\%$ & $97\%$ \\ \hline
        128x128x4-12 & $299\%$ & $274\%$ & $109\%$ \\ \hline
        128x128x8-12 & $424\%$ & $347\%$ & $122\%$ \\ \hline
    \end{tabular}
\end{center}

Average bench: $94.175$s

\subsection*{Stage 3 --- Division of Work}%
\label{sub:division-of-work}
\begin{center}
    \begin{tabular}{|c|c|c|c|}
        \hline
        \textbf{Benchmark} & \textbf{Baseline result} (ns/100 turns) &
        \textbf{Our result} & \textbf{\% Difference} \\ \hhline{|=|=|=|=|}
        128x128x2-12 & $736951655$ & $1730991762$ & $42\%$ \\ \hline
        128x128x4-12 & $537184229$ & $1264782083$ & $42\%$ \\ \hline
        128x128x8-12 & $363069928$ & $987918516$ & $36\%$ \\ \hline
    \end{tabular}
\end{center}
\begin{center}
    \begin{tabular}{|c|c|c|c|}
        \hline
        \textbf{Benchmark} & \textbf{Baseline CPU usage} &
        \textbf{Our CPU usage} & \textbf{\% Difference} \\ \hhline{|=|=|=|=|}
        128x128x2-12 & $184\%$ & $190\%$ & $96\%$ \\ \hline
        128x128x4-12 & $299\%$ & $274\%$ & $109\%$ \\ \hline
        128x128x8-12 & $425\%$ & $347\%$ & $122\%$ \\ \hline
    \end{tabular}
\end{center}

Average bench: $90.832$s

\subsection*{Stage 4 --- Cooperative Problem Solving}%
\label{sub:coop-solving}
\begin{table}[ht]
\caption{Benchmark comparison for Stage 4}
\begin{center}
    \begin{tabular}{|c|c|c|c|}
        \hline
        \textbf{Benchmark} & \textbf{Baseline result} (ns/100 turns) &
        \textbf{Our result} & \textbf{\% Difference} \\ \hhline{|=|=|=|=|}
        128x128x2-12 & $735620822$ & $716510865$ & $102\%$ \\ \hline
        128x128x4-12 & $530949831$ & $483555762$ & $109\%$ \\ \hline
        128x128x8-12 & $361530673$ & $333119799$ & $108\%$ \\ \hline
    \end{tabular}
\end{center}
\end{table}

\begin{table}[ht]
\caption{CPU usage for Stage 4}
\begin{center}
    \begin{tabular}{|c|c|c|c|}
        \hline
        \textbf{Benchmark} & \textbf{Baseline CPU usage} &
        \textbf{Our CPU usage} & \textbf{\% Difference} \\ \hhline{|=|=|=|=|}
        128x128x2-12 & $185\%$ & $289\%$ & $64\%$ \\ \hline
        128x128x4-12 & $299\%$ & $425\%$ & $70\%$ \\ \hline
        128x128x8-12 & $425\%$ & $667\%$ & $63\%$ \\ \hline
    \end{tabular}
\end{center}
\end{table}



Average bench: $41.679$s

\subsection*{Conclusions}%
\label{sub:Conclusions}
For smaller image sizes, the single thread solution outperforms the
initial divide and conquer method. This is because of the large overhead
cost of splitting the image up and reconstructing the world after each
turn.  However, for larger images, the divide and conquer algorithm does
improve on the times set by the single thread, because the cost of
reconstructing the world and splitting the threads is constant and does
not rely on the size of the image.

In general, the performance of the divide and conquer algorithm and the division of work algorithm is largely the same, especially for larger images. This is due to the fact that the only major difference between the two algorithms is that division of work algorithm allows the splitting of threads by multiples of 2, unlike the divide and conquer algorithm which splits the threads by powers of 2. This disparity between the two algorithms does not affect the performance of the algorithm and explains the minimal difference in the performance of the two solutions.

The only constant result we can find in these graphs is that the cooperative solving solution is consistently faster than any other algorithm. The Cooperative solving algorithms is always faster than either the divide and conquer solution or the division of work solution because the cooperative solving algorithm does not bother rebuilding the entire world after each turn, and instead passes the halo lines to the worker above or below it, while essentially doing the exact the same work as the other two algorithms. As for the single thread solution, the cooperative solving algorithm is still mostly faster because not having to reconstruct the world after each turn saves up a lot of time compared to workers just passing halo lines to workers above or below it. 

%TODO INSERT GRAPHS

\pgfplotstableread[col sep=comma]{16x16.csv}\datatable
\begin{figure}
    \begin{minipage}[b]{0.58\textwidth}
        \resizebox{\textwidth}{!}{%
\begin{tikzpicture}
\begin{axis}[
    legend style={
        at={(1,1)},
    anchor=north west,at={(axis description cs:1.1,0.7)}},
    xtick=data,
    title style={yshift=1.1ex,},
    title={\textbf{Comparison for 16x16 image}},
    xticklabels from table={\datatable}{threads},
    grid=both,
    xlabel={Picture size, threads and turns},
    ylabel={Operations per nanosecond},
]
\addplot table [col sep=comma, x expr=\coordindex, y=single, col sep=comma] {16x16.csv};
\addplot table [col sep=comma, x expr=\coordindex, y=divide-conquer, col sep=comma] {16x16.csv};
\addplot+[green] table [col sep=comma, x expr=\coordindex, y=coop, col sep=comma] {16x16.csv};
\addplot+[black] table [col sep=comma, x expr=\coordindex, y=division, col sep=comma] {16x16.csv};
    
\legend{Single, Divide and Conquer, Cooperative solving, Division of work}
\end{axis}
\end{tikzpicture}
}
\end{minipage}
\hfill
\begin{minipage}[b]{0.41\textwidth}
\resizebox{\textwidth}{!}{%
\pgfplotstableread[col sep=comma]{128x128.csv}\datatable
\begin{tikzpicture}
\begin{axis}[
    xtick=data,
    title style={yshift=1.1ex,},
    title={\textbf{Comparison for 128x128 image}},
    xticklabels from table={\datatable}{threads},
    grid=both,
    xlabel={Picture size, threads and turns},
    ylabel={Operations per nanosecond},
]
\addplot table [col sep=comma, x expr=\coordindex, y=single, col
    sep=comma] {128x128.csv};
\addplot table [col sep=comma, x expr=\coordindex, y=divide-conquer, col
    sep=comma] {128x128.csv};
\addplot+[green] table [col sep=comma, x expr=\coordindex, y=coop, col
    sep=comma] {128x128.csv};
\addplot+[black] table [col sep=comma, x expr=\coordindex, y=division, col
    sep=comma] {128x128.csv};
\end{axis}
\end{tikzpicture}
}
\end{minipage}
\end{figure}

\end{document}
