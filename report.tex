%%=====================================================================================
%%
%%       Filename:  report.tex
%%
%%    Description:  Report for our project
%%
%%        Version:  1.0
%%        Created:  02/12/19
%%       Revision:  none
%%
%%         Author:  YOUR NAME (), 
%%   Organization:  
%%      Copyright:  Copyright (c) 2019, YOUR NAME
%%
%%          Notes:  
%%
%%=====================================================================================
% Preamble {{{
\documentclass[11pt,a4paper,dvipsnames,cmyk]{scrartcl}
\usepackage[english]{babel}
\typearea{12}
%}}}

% Set indentation and line skip for paragraph {{{
\setlength{\parskip}{1em}
\usepackage[margin=2cm]{geometry}
\addtolength{\textheight}{-1in}
\setlength{\headsep}{.5in}
% }}}

\usepackage{hhline} 
\usepackage{mathtools} 
\usepackage[T1]{fontenc}
\usepackage[utf8]{inputenc}

% Headers setup {{{
\usepackage{fancyhdr}
\pagestyle{fancy}
\lhead{Game of Life}
\rhead{Josh Felmeden, Antoine Ritz}
\usepackage{hyperref} 
% }}}

% Listings {{{
\usepackage[]{listings,xcolor} 
\lstset
{
    breaklines=true,
    tabsize=3,
    showstringspaces=false
}

\definecolor{lstgrey}{rgb}{0.05,0.05,0.05}
\usepackage{listings}
\makeatletter
\lstset{language=[Visual]Basic,
    backgroundcolor=\color{lstgrey},
    frame=single,
    xleftmargin=0.7cm,
    frame=tlbr, framesep=0.2cm, framerule=0pt,
    basicstyle=\lst@ifdisplaystyle\color{white}\footnotesize\ttfamily\else\color{black}\footnotesize\ttfamily\fi,
    captionpos=b,
    tabsize=2,
    keywordstyle=\color{Magenta}\bfseries,
    identifierstyle=\color{Cyan},
    stringstyle=\color{Yellow},
    commentstyle=\color{Gray}\itshape
}
\makeatother
\renewcommand{\familydefault}{\sfdefault}
% }}}


% Other packages {{{
\usepackage{multicol}
\usepackage{needspace}
\usepackage{tcolorbox}
\usepackage{soul}
\usepackage{CormorantGaramond} 
%% The font package uses mweights.sty which has som issues with the
%% \normalfont command. The following two lines fixes this issue.
\let\oldnormalfont\normalfont
\def\normalfont{\oldnormalfont\mdseries}
\usepackage{textcomp}
\usepackage{amsmath} 
\usepackage{booktabs} 
\usepackage{tcolorbox} 
\usepackage[symbol]{footmisc} 
\renewcommand{\thefootnote}{\fnsymbol{footnote}}
\renewcommand{\familydefault}{\sfdefault}
% }}}

% Title {{{
\title{The Game of Life}
\subtitle{Concurrent Computing Coursework}
\author{Josh Felmeden, NK18044 \\ Antoine Ritz, }
% }}}

\begin{document}
\maketitle
\begin{multicols}{2}

\section*{Functionality and Design}%
\label{sec:func-and-design}
Our solution was built up by initially creating a single threaded solution
to the problem. This version iterates through the board bitwise, and for
each bit gathers all the `neighbours' for the cells (the 8 directly
adjacent cells). From this, the logic is applied and the cell is updated
if necessary. This is repeated for the desired number of turns.

From this, we created a multi-threaded solution. We split the board up
into strips and passed each strip to a worker. However, each worker would
also need information from the lines directly above and below its strip of
cells (called \textit{halo lines}). We decided to pass these halo lines
wrapped around the strips, so that the workers are able to calculate each
cell correctly. Once they have completed their strip, they return it to
the distributor function. The function reconstructs the world and begins
the process again for the desired number of turns.

One problem that we ran into was that we were passing the world by means
of pointer. This led to problems due to premature changes being applied to
the board. To solve this, we used channels to pass the board to the
workers.

The processing of the program is currently unable to be cancelled, and
therefore we added the ability to quit, pause processing, and show the
current state of the board with key presses. Alongside this, we also
implemented an output of the number of alive cells every two seconds using
a \textit{ticker}. 

Following this, we added the ability to allow the number of workers to
support all multiples of two, rather than powers of two alone. Initially,
this proved difficult, since at least one worker would receive a smaller
strip than the others, and consequently meant that some workers would
finish sooner than others. This resulted in a \textbf{deadlock} situation,
because the world reconstructing function expected all strips to be of
equal size, and therefore it was blocked when the final strip was smaller
than expected. To work around this, we made each worker apart from the
final one work on the same number of lines, and the final worker simply
had the remainder. Then, to reconstruct the world, we processed the final
worker output separately from the rest.

Finally, passing the entire world between each turn is time consuming,
because in reality, the only information that needs to be passed between
the workers are the \textit{halo lines}. Additionally, there is no reason
to reconstruct the entire world each turn. Implementing this proved to be
problematic, as sometimes the workers would get out of sync with one
another. To rectify this, we implemented a \textit{master} thread, which
served as a hard limit on the speed of the other threads to ensure that
the threads would not exceed one another. We passed in a new structure
with \texttt{masterTurns} via pointer, which contains the current turn of
the master thread.

A further problem was to implement the key presses %I DONT KNOW HOW I'M
%GOING TO DO THIS YET


%TODO: Text goes here

\newpage

\section*{Tests, Experiments, and Critical Analysis}%
\label{sec:experiment-and-analysis}
%TODO: Text goes here



\subsection*{Stage 1a --- Single Thread}%
\label{sub:single-thread}

\subsection*{Stage 1b --- Divide and Conquer}%
\label{sub:divide-conquer}

\subsection*{Stage 2a --- User Interaction}%
\label{sub:user-interaction}

\subsection*{Stage 2b --- Periodic Events}%
\label{sub:periodic-events}

\subsection*{Stage 3 --- Division of Work}%
\label{sub:division-of-work}

\subsection*{Stage 4 --- Cooperative Problem Solving}%
\label{sub:coop-solving}


\end{multicols}

\end{document}
